%! Author = kevin
%! Date = 2022-01-12

% Preamble
\documentclass[11pt]{article}
\title{Protein-Liganden-Docking}
\author{Kevin Kretz, German Esaulkov, Leander Schäfer}


% Packages
\usepackage{amsmath}
\usepackage{graphicx}


\renewcommand{\contentsname}{Inhaltsverzeichnis}
\graphicspath{ {media\_in\_doc/} }

% Document
\begin{document}

    \maketitle

    \tableofcontents


    \section{Kurzfassung}


    Tropenkrankheiten stellen in ihren Verbeitungsgebieten eine extreme Bedrohung für die dortige Bevölkerung dar. Gemessen an ihrer Bedeutung - Malaria ist z. B. mit 200 Millionen Fällen pro Jahr die häufigste Infektionskrankheit der Welt  - erfahren sie in nicht betroffenen Industrieländern nur wenig Aufmerksamkeit in Medien, in Form von Forschungsprojekten und in den Entwicklungsabteilungen von Pharmafirmen.

    Aufgrund der Veröffentlichung des AlphaFold-Papers im Juli 2021 und der gleichzeitig veröffentlichten Datenbank von dreidimensionalen Proteinmodellen, sowie dem UseGalaxy-Server der Universität Freiburg haben wir gute Voraussetzungen bekommen, um mit Hilfe von Protein-Liganden-Docking nach möglichen Wirkstoffen gegen Tropenkrankheiten zu suchen.


    % \section{Inhaltsverzeichnis}
    % \tableofcontents

    \section{Einleitung}

    Tropenkrankheiten stellen in Entwicklungsländern verheerende Schäden an und fordern viele Opfer ein. So ist z.B. Malaria mit 200 Millionen Fällen pro Jahr die häufigste Infektionskrankheit der Welt. Trotz der massiven Schäden, die Tropenkrankheiten, wie z.B. Malaria, die Afrikanische Schlafkrankheit und Hepatitis A- anrichten, schenken Industrieländer und die dort hiesigen Pharmakonzerne diesen Bedrohungen nur wenig Beachtung. Die Entwicklung und der Verkauf von Medikamenten gegen diese weit verbreiteten Krankheiten erschien den Konzernen als nicht lukrativ genug, obwohl der Markt dafür vorhanden wäre und wurde demnach vernachlässigt. (Doch mittlerweile gibt es Vereinigungen, die gegen diese reale Gefahr kämpfen. Wir möchten uns dem anschließen.)  Nun versuchen wir mit unserem Projekt mithilfe von Bioinformatik (weitere) helfende Arzneimittel zu finden und unter Umständen auch eine neue Entdeckung zu machen. Da uns keine Laborforschungsmittel oder Supercomputer zur Verfügung stehen, müssen wir auf öffentlich zugängliche Ressourcen zurückgreifen. Hierfür gibt es die webbasierte Plattform für Computational Science, mit Fokus auf Biologie, namens „Galaxy“. Damit kann man selbstverständlich kein fertiges Medikament entwickeln, dennoch erhoffen wir uns damit eine Grundlage für weitere Forschung zu schaffen. Dort versuchen wir mithilfe der zur Verfügung gestellten Werkzeuge Protein-Liganden- Docking zu simulieren und damit einen wirkungsvollen Stoff gegen die Krankheiten zu finden. Doch was genau ist dieses Protein-Liganden-Docking? Dieses Verfahren ist eine „molecular modelling“- Technik, wobei mittels Bioinformatik versucht wird herauszufinden, welche Liganden mit welchen Proteinen an welcher Stelle binden. Man benötigt dementsprechend Daten über den Liganden und den Rezeptor des Proteins. Jetzt hat dieses Verfahren eine pharmazeutische Bedeutung, da man zum einen mögliche Wirkstoffe finden kann, die vitale Proteine des Krankheitserregers außer Kraft setzen können. Weiter erleichtert es auch die Suche, da das Auswahlverfahren nun „in silico“ erfolgen kann und nicht alle möglichen Kandidaten im Labor getestet werden müssen.

    \includegraphics{Galaxy-logo.png}

    
    Auch wir möchten dieses bioinformatische Verfahren anwenden. Doch warum haben wir ausgerechnet diese Krankheiten gewählt? Malaria ist eine vor allem in den tropischen Regionen Afrikas anzutreffen, aber auch in Südostasien und in den nördlichen Teilen Südamerikas zu finden. Wie bereits erwähnt ist Malaria, auch Sumpf- oder Tropenfieber bekannt, die häufigste Tropenkrankheit. Der wichtigste Überträger der Krankheit ist die weibliche Anophelesmücke. Malaria kann Symptome wie Fieber, Erbrechen, Gelbsucht und Krämpfe umfassen. Vor allem die von uns gewählte Variante der Falciparum-Malaria ruft schwere Symptome wie Lähmung oder Koma hervor. Über Lungen- oder Nierenversagen führt die Krankheit zum Tod.

    Ebenfalls eine durch ein Insekt verbreitete Krankheit ist die Afrikanische Trypanosomiasis, oder auch Afrikanische Schlafkrankheit, dessen Erreger Trypanosoma brucei durch die Tsetse-Fliege übertragen wird. Man kann aktuell mit ca. 500 000 Betroffenen in Afrika rechnen. Über Fieber, Gliederschmerzen, Lymphknotenschwellung und Anämie führt die Krankheit zum namengebenden Dämmerzustand und anschließend zum Tod.

    Als dritte Krankheit betrachten wir die Chagas-Krankheit, auch Südamerikanische Trypanosomiasis genannt. Der Erreger Trypanosoma cruci, ein Verwandter der Trypanosoma brucei, wird durch den Kot verschiedener Raubwanzen, aber vor allem von Triatoma infestans, übertragen. Aktuell gibt es ca. 18 Millionen Erkrankte, wobei jedes Jahr 50.000 dazukommen. Zahl der Todesfälle beträgt jährlich um die 15.000. Die Krankheit verursacht Ödeme, chronisches Herzversagen und dem Absterben von Nervenzellen im Darm. Dies führt mitunter zum Tod durch Darmverschluss oder Darmdurchbruch.

    Sowohl die Chagas-Krankheit als auch die Afrikanische Trypanosomiasis sind von der Weltgesundheitsorganisation als „neglected tropical diseases“ (NTD´s) anerkannt.

    Mit Galaxy haben wir schon eine gute Grundlage zur Erforschung und Durchführung des Protein-Liganden-Dockings, jedoch fehlen uns die Daten über den Arzneistoff, also den Liganden, und das Organell, sprich ein vitales Protein, welches im Erreger außer Kraft gesetzt werden soll, sodass er stirbt. Man benötigt also Datenbanken, mit den entsprechenden Sequenzen bzw. Strukturen. Für das Protein gibt es die die Proteinstrukturdatenbank AlphaFold, welches von DeepMind und EMBL-EBI entwickelt wurde. Da werden Proteinstrukturen aufgeführt, die von einer AI auf Grundlage der Aminosäuresequenz ermittelt bzw. vorhergesagt wurden. Dabei sind diese Vorhersagen sehr präzise. Hat man dann eine mögliche Struktur für den Liganden gefunden, kann man nach realen Chemikalien in der EMBL-EBI Datenbank suchen. Das „European Molecular Biology Laboratory Bioinformatics Institute“ beherbergt die größte öffentlich zugängliche biologische Datenbank und bietet gleichzeitig auch bioinformatische Dienste für Forschende aus aller Welt an. Nur mithilfe von Galaxy, AlphaFold und EMBL-EBI können wir an unserem Projekt forschen. Und wie genau das abläuft wird im weiteren Verlauf erläutert.


    \section{Vorgehensweise, Materialien und Methode}

    Blabla


    \section{Ergebnisse}

    Blabla


    \section{Ergebnisdiskussion}

    Blabla


    \section{Zusammenfassung}

    Blabla


    \section{Quellen- und Literaturverzeichnis}

    Blabla


    \section{Unterstützungsleistungen}

    Blabla


\end{document}